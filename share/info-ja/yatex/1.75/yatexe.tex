\def\lang{jp} % -*- texinfo -*-
\input texinfo.tex
@setfilename yatexe
@settitle Yet Another tex-mode for Emacs

@iftex
@c @syncodeindex fn cp
@c Last modified Thu May 27 18:04:10 2010 on firestorm
@syncodeindex vr cp
@end iftex

@titlepage
@sp 10
@center
@subtitle Yet Another tex-mode for emacs
@title Wild Bird
@subtitle // YaTeX //
@author @copyright{} 1991-2003 by    HIROSE, Yuuji [yuuji@@yatex.org]
@end titlepage

@node Top, What is YaTeX?, (dir), (dir)
@comment  node-name,  next,  previous,  up
@cindex Demacs
@cindex Mule
@cindex LaTeX
@cindex YaTeX

@menu
* What is YaTeX?::              
* Main features::               What YaTeX can do
* Installation::                Guide to install
* Typesetting::                 Call typesetting processes
* %#notation::                  Meta-keyword `%#'
* Completion::                  Input LaTeX commands with completion
* Local dictionaries::          Directory dependent completion
* Commenting out::              Commenting/uncommenting text
* Cursor jump::                 Jumping to related position
* Changing and Deleting::       Changing/deleting certain unit of text
* Filling::                     Filling an item or paragraph
* Updation of includeonly::     Free from maintaining includeonly
* What column::                 Check what table-column the cursor belong
* Intelligent newline::         Guess requisites of new line
* Usepackage checker::          Selecting correct \usepackage is YaTeX's job
* Online help::                 On-line documentation of LaTeX
* Browsing file hierarchy::     Walking through file hierarchy
* Cooperation with other packages::  Work well with gmhist, min-out
* Customizations::              How to breed `Wild Bird'
* Etcetera::                    YaTeX is acquisitive.
* Copying::                     Redistribution


@node What is YaTeX?, Main features, Top, Top
@comment  node-name,  next,  previous,  up
@chapter What is YaTeX?

  YaTeX automates typesetting and previewing of LaTeX and enables
completing input of LaTeX mark-up command such as
@code{\begin@{@}}..@code{\end@{@}}.

  YaTeX also supports Demacs which runs on MS-DOS(386), Mule (Multi
Language Enhancement to GNU Emacs), and latex on DOS.

@node Main features, Installation, What is YaTeX?, Top
@comment  node-name,  next,  previous,  up
@chapter Main features

@itemize
@item Invocation of typesetter,  previewer and related programs(@kbd{C-c t})
@item Typesetting on static region which is independent from point
@item Semiautomatic replacing of @code{\includeonly}
@item Jumping to error line(@kbd{C-c '})
@item Completing-read of La@TeX{} commands such as @code{\begin@{@}},
        @code{\section} etc. 
        (@kbd{C-c b}, @kbd{C-c s}, @kbd{C-c l}, @kbd{C-c m})
@item Enclosing text into La@TeX{} environments or commands
      (@kbd{C-u} @var{AboveKeyStrokes})
@item Displaying the structure of text at entering sectioning commands
@item Lump shifting of sectioning commands (@ref{view-sectioning})
@item Learning unknown/new La@TeX{} commands for the next completion
@item Argument reading with a guide for complicated La@TeX{} commands
@item Generating argument-readers for new/unsupported commands(@file{yatexgen})
@item Quick changing or deleting of La@TeX{} commands(@kbd{C-c c}, @kbd{C-c k})
@item Jumping from and to inter-file, begin<->end, ref<->label(@kbd{C-c g})
@item Blanket commenting out or uncommenting
        (@kbd{C-c >}, @kbd{C-c <}, @kbd{C-c ,}, @kbd{C-c .})
@item Easy input of accent mark, math-mode's commands and Greek letters
        (@kbd{C-c a}, @kbd{;}, @kbd{:})
@item Online help for the popular La@TeX{} commands
      (@kbd{C-c ?}, @kbd{C-c /})
@item Document files hierarchy browser (@kbd{C-c d})
@item Adding automatically \usepackage corresponding to inputting LaTeX
      macro with completion
@item Allow you to forget creating \label{}s, \ref or \cite completion
      automatically generate labels.
@end itemize

@node Installation, Typesetting, Main features, Top
@comment  node-name,  next,  previous,  up
@chapter Installation
@cindex installation
@cindex .emacs
@cindex auto-mode-alist
@cindex autoload

  Put next two expressions into your @file{~/.emacs}.

@lisp
        (setq auto-mode-alist
              (cons (cons "\\.tex$" 'yatex-mode) auto-mode-alist))
        (autoload 'yatex-mode "yatex" "Yet Another La@TeX{} mode" t)
@end lisp

Next, add certain path name where you put files of YaTeX to your
load-path.  If you want to put them in @file{~/src/emacs}, write

@lisp
       (setq load-path
             (cons (expand-file-name "~/src/emacs") load-path))
@end lisp

@noindent
in your @file{~/.emacs}

  Then, yatex-mode will be automatically loaded when you visit a
file which has extension @file{.tex}.  If yatex-mode is successfully
loaded, mode string on mode line will be turned to "YaTeX".


@node Typesetting, %#notation, Installation, Top
@comment  node-name,  next,  previous,  up
@chapter Typesetting
@cindex typesetting
@cindex previewer
@cindex typesetter
@cindex latex
@cindex printing out

  The prefix key stroke of yatex-mode is @kbd{C-c} (Press 'C' with Control
key) by default.  If you don't intend to change the prefix key stroke,
assume all @kbd{[prefix]} as @kbd{C-c} in this document.  These key
strokes execute typeset or preview command.

@table @kbd
@item [prefix] t j
        @dots{} invoke latex
@item [prefix] t r
        @dots{} invoke latex on region
@item [prefix] t e
        @dots{} invoke latex on current environment or whole
         portion of current formulas in math-mode.
@item [prefix] t k
        @dots{} kill current typesetting process
@item [prefix] t b
        @dots{} invoke bibtex
@item [prefix] t i
        @dots{} invoke makeindex
@item [prefix] t d
        @dots{} invoke latex && dvipdfmx
@item [prefix] t p
        @dots{} preview
@item [prefix] t l
        @dots{} lpr dvi-file
@item [prefix] t s
        @dots{} search current string on xdvi-remote
@end table

@menu
* Calling typesetter::          
* Calling previewer::           
* Printing out::                
@end menu

@node Calling typesetter, Calling previewer, Typesetting, Typesetting
@comment  node-name,  next,  previous,  up
@section Calling typesetter

  Typing @kbd{[prefix] t j}, the current editing window will be divided
horizontally when you invoke latex command, and log message of La@TeX{}
typesetting will be displayed in the other window; called typesetting
buffer.  The typesetting buffer automatically scrolls up and traces
La@TeX{} warnings and error messages.  If you see latex stopping by an
error, you can send string to latex in the typesetting buffer.

  If an error  stops the La@TeX{}  typesetting, this  key stroke will
move the cursor to the line where La@TeX{} error is detected.

@table @kbd
@item [prefix] '
@itemx ([prefix]+single quotation)

        @dots{} jump to the previous error or warning
@end table

  If you find a noticeable error, move to the typesetting buffer and move
the cursor on the line of error message and type @kbd{SPACE} key.  This
makes the cursor move to corresponding source line.

  YaTeX-typeset-region invoked by @kbd{[prefix] tr} call typesetter
for region.  The region is specified by standard point and mark, or
by @code{%#BEGIN} and @code{%#END} marks.  Selected region will be
copied to the temporary file @file{texput.tex} with the same preamble
as the main file of current editing sources.  Be sure to put
all local macro settings in preamble, not after @code{\begin{document}}.
  The method of specification of the region is shown in the
section @xref{%#notation}.

  The documentclass  for typeset-region is the same as that of editing
file if you edit one  file,  and is the same as main file's if you
edit splitting files.

  The @kbd{[prefix] te} key automatically marks current inner environment
or inner math mode and then call typeset-region with marked region.  This
is convenient to quick view of current tabular environment or current
editing formulas.  Keeping previewer window for @file{texput.dvi} is handy
for debugging.  Since @kbd{[prefix] te} selects the inner-most environment
as region, it is not suitable for partial typesetting of doubly or more
composed environment.  If you want to do partial typesetting for a nested
environment, use @kbd{[prefix] tr} for static-region, which is described
in the section @xref{%#notation}.

@node Calling previewer, Printing out, Calling typesetter, Typesetting
@comment  node-name,  next,  previous,  up
@section Calling previewer

  @kbd{[prefix] t p} invokes the TeX previewer.  And if you are using
xdvi-remote, which can be controled from other terminals, @kbd{[prefix] t
s} enables you to search current string at the cursor on the running xdvi
window.

@node Printing out,  , Calling previewer, Typesetting
@comment  node-name,  next,  previous,  up
@section Printing out

  When you type @code{[preifx] t l}, YaTeX asks you the range of
dvi-printing by default.  You can skip this by invoking it with
universal-argument as follows:

@example
        C-u [prefix] tl
@end example

@node %#notation, Completion, Typesetting, Top
@comment  node-name,  next,  previous,  up
@chapter %# notation
@cindex %# notation

  You can control the typesetting process by describing @code{%#}
notations in the source text.

@menu
* Changing typesetter::         
* Splitting input files::       
* Static region for typesetting::  
* Lpr format::                  
* Editing %# notation::         
@end menu

@node Changing typesetter, Splitting input files, %#notation, %#notation
@comment  node-name,  next,  previous,  up
@section To change the `latex' command or to split a source text.
@cindex typesetter

  To change the typesetting command, write

@example
        %#!latex-big
@end example

@noindent
anywhere in the source text.  This is useful for changing
typesetter.

@node Splitting input files, Static region for typesetting, Changing typesetter, %#notation
@comment  node-name,  next,  previous,  up
@section Splitting input files

  And if you split the source text and
edit subfile that should be included from main text.

@example
        %#!latex main.tex
@end example

@noindent
will be helpful to execute latex on main file from sub text buffer.  Since
this command line after @kbd{%#!} will be sent to shell literally, next
description makes it convenient to use ghostview as dvi-previewer.

@example
        %#!latex main ; dvi2ps main.dvi > main
@end example

@noindent
Note that YaTeX  assumes the component  before the  last period of
the last word in this line as base name of the main La@TeX{} source.
The @code{%f} notation in this line is replaced by main file name, and
@code{%r} replaced by root name of main file name.  If you specify
@code{%f} or @code{%r}, YaTeX always ask you the name of main file at the
first typesetting.

  To make best use of the feature of inter-file jumping by
@kbd{[prefix] g} (see @ref{Cursor jump}), take described below into
consideration.

@itemize
@item You can put split texts in sub directory, but not in
 sub directory of sub directory.
@item In the main text, specify the child file name with relative path name
 such as \include@{chap1/sub@}, when you include the file in
 a sub-directory.
@item In a sub-text, write @code{%#!latex main.tex} even if @file{main.tex}
 is in the parent directory(not %#!latex ../main.tex).
@end itemize

@node Static region for typesetting, Lpr format, Splitting input files, %#notation
@comment  node-name,  next,  previous,  up
@section Static region
@cindex static region
@cindex Fixed region

  Typeset-region by @kbd{[prefix] tr} passes the region between point and
mark to typesetting command by default.  But when you want to typeset
static region, enclose the region by @code{%#BEGIN} and @code{%#END} as
follows.

@example
        %#BEGIN
          TheRegionYouWantToTypesetManyTimes
        %#END
@end example

This is the rule of deciding the region.

@enumerate
@item
If there exists %#BEGIN before point,

@enumerate
@item
If there exists %#END after %#BEGIN,
@itemize
@item From %#BEGIN to %#END.
@end itemize

@item
If %#END does not exist after %#BEGIN,
@itemize
@item From %#BEGIN to the end of buffer.
@end itemize
@end enumerate

@item
If there does not exist %#BEGIN before point,
@itemize
@item Between point and mark(standard method of Emacs).
@end itemize
@end enumerate

  It is useful to write @code{%#BEGIN} in the previous line of \begin and
@code{%#END} in the next line of \@code{end} when you try complex
environment such as `tabular' many times.  It is also useful to put only
@code{%#BEGIN} alone at the middle of very long text.  Do not forget to
erase @code{%#BEGIN} @code{%#END} pair.

@node Lpr format, Editing %# notation, Static region for typesetting, %#notation
@comment  node-name,  next,  previous,  up
@section Lpr format
@cindex lpr format

  Lpr format is specified by three Lisp variables.  Here are the
default values of them.

@table @code
@item (1)dviprint-command-format
        @code{"dvi2ps %f %t %s | lpr"}
@item (2)dviprint-from-format
        @code{"-f %b"}
@item (3)dviprint-to-format
        @code{"-t %e"}
@end table

  On YaTeX-lpr, @code{%s} in (1) is replaced by the file name of main
text, @code{%f} by contents of (2), %t by contents of (3).  At these
replacements, @code{%b} in (2) is also replaced by the number of beginning
page, @code{%e} in (3) is replaced by the number of ending page.  But
@code{%f} and @code{%t} are ignored when you omit the range of print-out
by @kbd{C-u [prefix] tl}.

  If you want to change this lpr format temporarily, put a command
such as follows somewhere in the text:

@example
        %#LPR dvi2ps %f %t %s | 4up -page 4 | texfix | lpr -Plp2
@end example

  And if you want YaTeX not to ask you the range of printing
out, the next example may be helpful.

@example
        %#LPR dvi2ps %s | lpr
@end example

@node Editing %# notation,  , Lpr format, %#notation
@comment  node-name,  next,  previous,  up
@section Editing %# notation

  To edit @code{%#} notation described above, type

@table @kbd
@item [prefix] %
        @dots{} editing %# notation menu
@end table

@noindent
and select one of the entry of the menu as follows.

@example
        !)Edit-%#! B)EGIN-END-region L)Edit-%#LPR
@end example

@noindent
Type @kbd{!} to edit @code{%#!} entry, @code{b} to enclose the region with
@code{%#BEGIN} and @code{%#END}, and @code{l} to edit @code{%#LPR} entry.
When you type @kbd{b}, all @code{%#BEGIN} and @code{%#END} are
automatically erased.

@node Completion, Local dictionaries, %#notation, Top
@comment  node-name,  next,  previous,  up
@chapter Completion
@cindex completion

  YaTeX makes it easy to input the La@TeX{} commands.  There are several
kinds of completion type, begin-type, section-type, large-type, etc...

@menu
* Begin-type completion::       
* Section-type completion::     
* Large-type completion::       
* Maketitle-type completion::   
* Arbitrary completion::        
* End completion::              
* Accent completion::           
* Image completion::            
* Greek letters completion::    
@end menu

@node Begin-type completion, Section-type completion, Completion, Completion
@comment  node-name,  next,  previous,  up
@section Begin-type completion
@cindex begin-type completion
@cindex environment
@cindex prefix b

  "Begin-type completion" completes commands of @code{\begin@{env@}} ... 
@code{\end@{env@}}.  All of the begin-type completions begin with this key
sequence.

@table @kbd
@item [prefix] b
        @dots{} start begin-type completion
@end table

@noindent
An additional key  stroke immediately  completes a frequently used
La@TeX{} @code{\begin@{@}}...@code{\@code{end}@{@}} environment.

@table @kbd
@item [prefix] b c
        @dots{}  @code{\begin@{center@}...\end@{center@}}
@item [prefix] b d
        @dots{}  @code{\begin@{document@}...\end@{document@}}
@item [prefix] b D
        @dots{}  @code{\begin@{description@}...\end@{description@}}
@item [prefix] b e
        @dots{}  @code{\begin@{enumerate@}...\end@{enumerate@}}
@item [prefix] b E
        @dots{}  @code{\begin@{equation@}...\end@{equation@}}
@item [prefix] b i
        @dots{}  @code{\begin@{itemize@}...\end@{itemize@}}
@item [prefix] b l
        @dots{}  @code{\begin@{flushleft@}...\end@{flushleft@}}
@item [prefix] b m
        @dots{}  @code{\begin@{minipage@}...\end@{minipage@}}
@item [prefix] b t
        @dots{}  @code{\begin@{tabbing@}...\end@{tabbing@}}
@item [prefix] b T
        @dots{}  @code{\begin@{tabular@}...\end@{tabular@}}
@item [prefix] b^T
        @dots{}  @code{\begin@{table@}...\end@{table@}}
@item [prefix] b p
        @dots{}  @code{\begin@{picture@}...\end@{picture@}}
@item [prefix] b q
        @dots{}  @code{\begin@{quote@}...\end@{quote@}}
@item [prefix] b Q
        @dots{}  @code{\begin@{quotation@}...\end@{quotation@}}
@item [prefix] b r
        @dots{}  @code{\begin@{flushright@}...\end@{flushright@}}
@item [prefix] b v
        @dots{}  @code{\begin@{verbatim@}...\end@{verbatim@}}
@item [prefix] b V
        @dots{}  @code{\begin@{verse@}...\end@{verse@}}
@end table

  Any other La@TeX{} environments are made by  completing-read of the
Emacs function.

@table @kbd
@item [prefix] b SPACE
        @dots{} begin-type completion
@end table

@noindent
The next message will show up in the minibuffer

@example
        Begin environment(default document): 
@end example

@noindent
by typing @kbd{[prefix] b}.  Put the wishing environment with completion
in the minibuffer, and @code{\begin@{env@}}...\@code{\end@{env@}} will be
inserted in the La@TeX{} source text.  If the environment you want to put
does not exist in the YaTeX completion table, it will be registered in the
user completion table.  YaTeX automatically saves the user completion
table in the user dictionary file at exiting of emacs.

At the completion of certain environments, the expected initial entry will 
automatically inserted such as @code{\item} for @code{itemize}
environment.  If you don't want the entry, it can be removed by undoing.

  If you want to  enclose some paragraphs  which have already been
written, invoke the  begin-type completion with changing  the case
of @kbd{b} of key sequence upper(or invoke it with  universal argument
by @kbd{C-u} prefix).
@cindex enclose region into environment

  The following example encloses a region with `description'
environment.

@table @kbd
@item [prefix] B D
@itemx (or ESC 1 [prefix] b D)
@itemx (or  C-u  [prefix] b D)

        @dots{} begin-type completion for region
@end table

  This enclosing holds good for the completing input by @kbd{[prefix] b
SPC}.  @kbd{[prefix] B SPC} enclose a region with the environment selected
by completing-read.

@node Section-type completion, Large-type completion, Begin-type completion, Completion
@comment  node-name,  next,  previous,  up
@section Section-type completion
@cindex section-type completion
@cindex prefix s

  "Section-type completion" completes section-type commands which take an
argument or more such as @code{\section@{foo@}}.  To invoke section-type
completion, type

@table @kbd
@item [prefix] s
        @dots{} section-type completion
@end table

@noindent
then the prompt

@example
        (C-v for view) \???@{@} (default documentclass):
@end example

@noindent
will  show up in the  minibuffer.  Section-type La@TeX{} commands are
completed by space key, and the default value is selected when you
type nothing in the minibuffer.

  Next, 

@example
        \section@{???@}:
@end example

@noindent
prompts you the argument of section-type La@TeX{} command.  For
example, the following inputs

@example
        \???@{@} (default documentclass): section
        \section@{???@}: Hello world.
@end example

@noindent
will insert the string

@example
        \section@{Hello world.@}
@end example

in your La@TeX{} source.  When you neglect argument such as

@example
        (C-v for view) \???@{@} (default section): vspace*
        \vspace*@{???@}: 
@end example

YaTeX puts

@example
        \vspace*@{@}
@end example

@noindent
and move the cursor in the braces.

  In La@TeX{} command, there are commands which take more than one
arguments such as @code{\addtolength@{\topmargin@}@{8mm@}}.  To complete these
commands, invoke section-type completion with universal argument as,
@cindex number of argument

@example
        C-u 2 [prefix] s (or ESC 2 [prefix] s)
@end example

@noindent
and make answers in minibuffer like this.

@example
        (C-v for view) \???@{@} (default vspace*): addtolength
        \addtolength@{???@}: \topmargin
        Argument 2: 8mm
@end example

@code{\addtolength} and the first argument @code{\topmargin} can be typed
easily by completing read.  Since YaTeX also learns the number of
arguments of section-type command and will ask that many arguments in
future completion, you had better tell the number of arguments to YaTeX at
the first completion of the new word.  But you can change the number of
arguments by calling the completion with different universal argument
again.


  Invoking section-type completion with @code{[Prefix] S} (Capital `S')
includes the region as the first argument of section-type command.

  The  section/large/maketitle type completion  can  work at the
prompt for   the argument   of  other section-type   completion.
Nested La@TeX{}  commands are  efficiently read with  the recursive
completion by typing  YaTeX's   completion key sequence in   the
minibuffer.

@menu
* view-sectioning::             
@end menu

@node view-sectioning,  , Section-type completion, Section-type completion
@comment  node-name,  next,  previous,  up
@subsection view-sectioning
@cindex view sectioning
@cindex outline

  In the minibuffer at the prompt of section-type command completion,
typing @kbd{C-v} shows a list of sectioning commands in source text(The
line with @code{<<--} mark is the nearest sectioning command).  Then,
default sectioning command appears in the minibuffer.  You can go up/down
sectioning command by typing @kbd{C-p}/@kbd{C-n}, can scrolls up/down the
listing buffer by @kbd{C-v}/@kbd{M-v}, and can hide sectioning commands
under certain level by 0 through 6.  Type @kbd{?}  in the minibuffer of
sectioning prompt for more information.

  You can generate this listing buffer (@code{*Sectioning Lines*} buffer)
by typing
@table @kbd
@item M-x YaTeX-section-overview
        @dots{} Generate *Sectioning Lines* buffer
@end table
@cindex{Generate the listing of sectioning units}
from the LaTeX source buffer.  In this listing buffer, typing @kbd{u} on
the sectioning command shifts up the corresponding sectioning command in
source text and @kbd{d} shifts down.  After marking lines in the listing
buffer, typing @kbd{U} shifts up all sectioning commands in the region,
and @kbd{U} shifts down.  Here are all the key bindings of
@code{*Sectioning Lines*} buffer.

@table @kbd
@item SPC
        @dots{} Jump to corresponding source line
@item .
        @dots{} Display corresponding source line
@item u
        @dots{} Shift up a sectioning line
@item d
        @dots{} Shift down a sectioning line
@item U
        @dots{} Shift up sectioning lines in region
@item D
        @dots{} Shift down sectioning lines in region
@item 0@dots{}6
        @dots{} Hide sectioning commands whose level is lower than n
@end table


@node Large-type completion, Maketitle-type completion, Section-type completion, Completion
@comment  node-name,  next,  previous,  up
@section Large-type completion

  "Large-type   completion"  inputs  the  font  or  size  changing
descriptions such as @code{@{\large @}}.  When you type

@table @kbd
@item [prefix] l
        @dots{} large-type completion
@end table

@noindent
the message in the minibuffer

@example
        @{\??? @} (default large): 
@end example

prompts prompts you large-type command with completing-read.  There are
TeX commands to change fonts or sizes, @code{it}, @code{huge} and so on,
in the completion table.

  Region-based completion is also invoked by changing the letter after
prefix key stroke as @kbd{[prefix] L}.  It encloses the region by braces
with large-type command.

@node Maketitle-type completion, Arbitrary completion, Large-type completion, Completion
@comment  node-name,  next,  previous,  up
@section Maketitle-type completion
@cindex maketitle-type completion

  We call it "maketitle-type completion" which completes commands such as
@code{\maketitle}.  Take notice that maketitle-type commands take no
arguments.  Then, typing

@table @kbd
@item [prefix] m
        @dots{} maketitle-type completion
@end table

@noindent
begins maketitle-completion.  Above mentioned  method is  true for
maketitle-completion, and   there  are   La@TeX{} commands    with no
arguments in completion table.

@node Arbitrary completion, End completion, Maketitle-type completion, Completion
@comment  node-name,  next,  previous,  up
@section Arbitrary completion
@cindex arbitrary completion

@noindent
  You can complete certain La@TeX{} command anywhere without typical
completing method as described, by typing

@table @kbd
@item [prefix] SPC
        @dots{} arbitrary completion
@end table

@noindent
after the initial string of La@TeX{} command that is preceded by @code{\}.

@node End completion, Accent completion, Arbitrary completion, Completion
@comment  node-name,  next,  previous,  up
@section End completion
@cindex end completion

@noindent
  YaTeX automatically detects the opened environment and close it with
\@code{\end@{environment@}}.  Though proficient YaTeX users never fail to
make environment with begin-type completion, some may begin an environment
manually.  In that case, type

@table @kbd
@item [prefix] e
        @dots{} @code{end} completion
@end table

@noindent
at the end of the opened environment.

@node Accent completion, Image completion, End completion, Completion
@comment  node-name,  next,  previous,  up
@section Accent completion
@cindex accent completion

  When you want to write the European accent marks(like @code{\`@{o@}}),

@table @kbd
@item [prefix] a
        @dots{} accent completion
@end table

@noindent
shows the menu

@example
        1:` 2:' 3:^ 4:" 5:~ 6:= 7:. u v H t c d b
@end example

@noindent
in the minibuffer.  Chose one character or corresponding numeric,
and you will see

@example
        \`@{@}
@end example

@noindent
in the editing buffer with the cursor positioned  in braces.  Type
one more character `o' for example, then

@example
        \`@{o@}
@end example

@noindent
will be completed, and the cursor gets out from braces.

@node Image completion, Greek letters completion, Accent completion, Completion
@comment  node-name,  next,  previous,  up
@section Image completion of mathematical sign
@cindex image completion
@cindex math-mode
@cindex sigma
@cindex leftarrow
@cindex ;

  Arrow  marks,  sigma mark and those signs mainly used  in  the
TeX's  math environment  are completed by  key  sequences  which
imitate the  corresponding symbols graphically.  This completion
only works in the math environment.  YaTeX automatically detects
whether the  cursor  located  in math environment  or  not,  and
change the behavior of key strokes @kbd{;} and @kbd{:}.

  By the way, we often express the leftarrow mark by `<-' for example.
Considering such image, you can write @code{\leftarrow} by typing @kbd{<-}
after @kbd{;} (semicolon) as a prefix.  In the same way,
@code{\longleftarrow} (@code{<--}) is completed by typing @kbd{;<--},
infinity mark which is imitated by @code{oo} is completed by typing
@kbd{;oo}.

  Here are the sample operations in YaTeX math-mode.

@example
INPUT                   Completed La@TeX{} commands
; < -                   @code{\leftarrow}
; < - -                 @code{\longleftarrow}
; < - - >               @code{\longleftrightarrow}
; o                     @code{\circ}
; o o                   @code{\infty}
@end example

  In  any case, you can quit  from image completion and can move
to the next editing  operation if the La@TeX{}  command you want is
shown in the buffer.

  @code{;} itself in math-environment is inserted by @kbd{;;}.  Typing
@kbd{TAB} in the midst of image completion shows all of the La@TeX{}
commands that start with the same name as string you previously typed in.
In this menu buffer, press @kbd{RET} after moving the cursor (by @kbd{n},
@kbd{p}, @kbd{b}, @kbd{f}) to insert the La@TeX{} command.

  To know all of the completion table, type @kbd{TAB} just after @kbd{;}.
And here is the sample menu by @kbd{TAB} after @kbd{;<}.

@example
KEY             LaTeX sequence          sign
<               \leq                    <
                                        ~
<<              \ll                     << 
<-              \leftarrow              <-
<=              \Leftarrow              <=
@end example

  You can define your favorite key-vs-sequence completion table in the
Emacs-Lisp variable @code{YaTeX-math-sign-alist-private}.  See also
@file{yatexmth.el} for the information of the structure of this variable.

@node Greek letters completion,  , Image completion, Completion
@comment  node-name,  next,  previous,  up
@section Greek letters completion
@cindex Greek letters completion
@cindex :

  Math-mode of YaTeX provides another image completion, Greek letters
completion in the same method.  After prefix @kbd{:}, typing @kbd{a} makes
@code{\alpha}, @kbd{b} makes @code{\beta} and @kbd{g} makes @code{\gamma}
and so on.  First, type @kbd{:TAB} to know all the correspondence of
alphabets vs. Greek letters.

  If you will find @kbd{;} or @kbd{:} doesn't work in correct position of
math environment, it may be a bug of YaTeX.  Please send me a bug report
with the configuration of your text, and avoid it temporarily by typing
@kbd{;} or @kbd{:} after universal-argument(@kbd{C-u}) which forces
@kbd{;} and @kbd{:} to work as math-prefix.

@node Local dictionaries, Commenting out, Completion, Top
@comment  node-name,  next,  previous,  up
@chapter Local dictionaries
@cindex local dictionaries
@cindex nervous users

  Tables for completion consist of three dictionaries; `standard
dictionary' built in @file{yatex.el}, `user dictionary' for your common
private commands, and `local dictionary' that is effective in a certain
directory.

  When you input the command unknown to YaTeX at a completion in the
minibuffer, YaTeX asks you with the following prompt;

@example
  `foo' is not in table. Register into: U)serDic L)ocalDic N)one D)iscard
@end example

@noindent 
In this menu, typing @kbd{u} updates your `user dictionary', @kbd{l}
updates your local dictionary, @kbd{n} updates only on-memory dictionary
which go through only current Emacs session, and @kbd{d} updates no
dictionary and throws the new word away.

  If you find this switching feature meaningless and bothersome, put the
next expression into your @file{~/.emacs}

@lisp
        (setq YaTeX-nervous nil)
@end lisp

@node Commenting out, Cursor jump, Local dictionaries, Top
@comment  node-name,  next,  previous,  up
@chapter Commenting out
@cindex commenting out
@cindex prefix >
@cindex prefix <
@cindex prefix ,
@cindex prefix .

  You may want to comment out some region.

@table @kbd
@item [prefix] >
        @dots{} comment out region by %
@item [prefix] <
        @dots{} uncomment region
@end table

@noindent
cause an operation to the region between point and mark.

@table @kbd
@item [prefix] .
        @dots{} comment out current paragraph
@item [prefix] ,
        @dots{} uncomment current paragraph
@end table

@noindent
comments or  uncomments the paragraph  where the  cursor  belongs.
This  `paragraph' means   the   region marked    by  the  function
mark-paragraph,  bound    to  @kbd{ESC h}   by   default.   It  is NOT
predictable  what will happen  when you  continuously  comment out
some paragraph many times.

  You can also comment out an environment between @code{\begin} and
@code{\end}, or a @code{\begin}-\@code{\end} pair themselves, by making the
following key strokes on the line where @code{\begin@{@}} or
@code{\end@{@}} exists.

@table @kbd
@item [prefix] >
        @dots{} comment out from \begin to \@code{end}
@item [prefix] <
        @dots{} uncomment from \begin to \@code{end}
@end table

@noindent
comment whole the contents of environment.  Moreover,

@table @kbd
@item [prefix] .
        @dots{} comment out \begin and \@code{end}
@item [prefix] ,
        @dots{} uncomment \begin and \@code{end}
@end table

@noindent
(un)comments out only environment declaration: @code{\begin@{@}} and
@code{\end@{@}}.  NOTE that even if you intend to comment out some region,
invoking @kbd{[prefix] >} on the @code{\begin},@code{\end} line decides to
work in `commenting out from @code{\begin} to @code{\end}' mode.


@node Cursor jump, Changing and Deleting, Commenting out, Top
@comment  node-name,  next,  previous,  up
@chapter Cursor jump
@cindex cursor jump
@cindex prefix g


@menu
* Jump to corresponding object::  
* Invoking image processor::    
* Jump to main file::           
* Jumping around the environment::  
* Jumping to last completion position::  
@end menu

@node Jump to corresponding object, Invoking image processor, Cursor jump, Cursor jump
@comment  node-name,  next,  previous,  up
@section Jump to corresponding object

  Typing

@table @kbd
@item [prefix] g
        @dots{} go to corresponding object
@end table

@noindent
in a certain place move the cursor to the place corresponding to the
La@TeX{} command of last place.  YaTeX recognize the followings as pairs
that have relation each other.

@itemize @bullet
@item @code{\begin@{@}} <-> @code{\end@{@}}
@item @code{%#BEGIN} <-> @code{%#END}
@item On the image-including line -> corresponding viewer or drawing tool
@item @code{\label@{@}} <-> @code{\ref@{@}}
@item @code{\include(\input)} -> included file
@item @code{\bibitem@{@}} <-> @code{\cite@{@}}
@end itemize

  On a @code{\begin},@code{\end} line, typing @kbd{[prefix] g} moves the
cursor to the corresponding @code{\end},@code{\begin} line, if its partner
really exists.  The behavior on the line @code{%#BEGIN} and @code{%#END}
are the same.  Note that if the correspondent of @code{label/ref} or
@code{cite/bibitem} exists in another file, that file have to be opened to
make a round trip between references by @kbd{[prefix] g}.

  If you type @code{[prefix] g} on the line of @code{\include@{chap1@}},
typically in the main text, YaTeX switches buffer to @file{chap1.tex}.

@table @kbd
@item [prefix] 4 g
        @dots{} go to corresponding object in other window
@end table

@noindent
do the same job as @kbd{[prefix] g} except it's done in other window.
Note that this function doesn't work on @code{begin/end},
@code{%#BEGIN/%#END} pairs because it is meaningless.

@node Invoking image processor, Jump to main file, Jump to corresponding object, Cursor jump
@comment  node-name,  next,  previous,  up
@section Invoking image processor
@cindex{Drawing tool invocation}

`image-including line' described above means such lines as
@code{\epsfile@{file=foo.ps@}}.  If you type @kbd{[prefix] g} on that
line, YaTeX automatically searches source of `foo.ps' and invokes image
viewer or drawing tool correspoinding to it.  For example; if you draw
an image foo.obj with Tgif and enclose its product named foo.eps by
@code{\epsfile} command.  Typing @kbd{[prefix] g} on @code{\epsfile} line
make YaTeX invoke @code{tgif foo.obj}.  How a processor is choosen is as
follows.

@enumerate
@item
If there is an expression matching with one of the pattern
defined in @code{YaTeX-processed-file-regexp-alist}, extract file name
from regexp group surrounded by \\(\\).  (Which group corresponds is
written in the cdr part of each list.)  If no matches were found, do
nothing.
@item
If there is a pattern as `%PROCESSOR' which is defined in the variable
@code{YaTeX-file-processor-alist}, call that processor giving the
file name with corresponding extension.
@item
If not, check the existence of each file which is supplied the
extension in the cdr part of each list of
@code{YaTeX-file-processor-alist}.  If any, call the corresponding
image viewer or drawing tool.
@end enumerate

@node Jump to main file, Jumping around the environment, Invoking image processor, Cursor jump
@comment  node-name,  next,  previous,  up
@section Jump to main file

  Typing

@table @kbd
@item [prefix] ^
        @dots{} visit main file
@item [prefix] 4^
        @dots{} visit main file in other buffer
@end table
@cindex prefix ^
@cindex prefix 4 ^

in a sub text switch the buffer to the main text specified by
@code{%#!}  notation.

@node Jumping around the environment, Jumping to last completion position, Jump to main file, Cursor jump
@comment  node-name,  next,  previous,  up
@section Jumping around the environment

  And these are the functions which work on the current La@TeX{}
environment:

@table @kbd
@item M-C-a
        @dots{} beginning of environment
@item M-C-e
        @dots{} @code{end} of environment
@item M-C-@@
        @dots{} mark environment
@end table
@cindex M-C-a
@cindex M-C-e
@cindex M-C-@@

@node Jumping to last completion position,  , Jumping around the environment, Cursor jump
@comment  node-name,  next,  previous,  up
@section Jumping to last completion position

YaTeX always memorize the position of completion into register @code{3}.
So every time you make a trip to any other part of text other than you are
writing, you can return to the editing paragraph by calling
register-to-point with argument YaTeX-current-position-register, which is
achieved by typing @kbd{C-x j 3}(by default).

@node Changing and Deleting, Filling, Cursor jump, Top
@comment  node-name,  next,  previous,  up
@chapter Changing and Deleting

  These  functions  are for change or   deletion of La@TeX{} commands
already entered.

@table @kbd
@item [prefix] c
        @dots{} change La@TeX{} command
@item [prefix] k
        @dots{} kill La@TeX{} command
@end table
@cindex prefix c
@cindex prefix k

@menu
* Changing LaTeX commands::     
* Killing LaTeX commands::      
@end menu

@node Changing LaTeX commands, Killing LaTeX commands, Changing and Deleting, Changing and Deleting
@comment  node-name,  next,  previous,  up
@section Changing La@TeX{} commands

@kbd{[prefix] c} can change the various (La)@TeX{} commands.  This can
change the followings.
@itemize @bullet
@item Environment names
@item Section-type commands
@item Argument of section-type commands
@item Optional parameters (enclosed by []) of section-type commands
@item Font/size designators
@item Math-mode's maketitle-type commands that can be inputted with
image completion
@end itemize

  Typing @kbd{[prefix] c} on one of above objects you want to change
brings a suitable reading function sometimes with completion.
Note: If you want to change the argument of section-type command that
contains other La@TeX{} commands, type @kbd{[prefix] c} either of
surrounding braces of the argument in order to make YaTeX ignore the
internal La@TeX{} sequences as an object of changing.  Anyway, it is
very difficult to know which argument position the cursor belongs because
the La@TeX{} commands can be nested and braces can freely emerge.  So keep 
it mind to put the cursor on a brace when you are thinking of changing a
complicated argument.

@node Killing LaTeX commands,  , Changing LaTeX commands, Changing and Deleting
@comment  node-name,  next,  previous,  up
@section Killing La@TeX{} commands
@cindex Killing La@TeX{} commands

  @kbd{[prefix] k} kills the La@TeX{} commands sometimes with their
arguments.  Following table illustrates the correspondence of the invoking
position and what is killed.

@example
[Invoking position]             [action]
\begin, \end line               kill \begin,\end pairs
%#BEGIN, %#END line             kill %#BEGIN,%#END pairs
on a Section-type command       kill section-type command
on a parenthesis                kill parentheses
@end example

Note that when killing @code{\begin, \end} or @code{%#BEGIN, %#END} pair,
the lines @code{\begin, \end} or @code{%#BEGIN, %#END} exist will be
killed entirely.  So take care not to create any line that contains more
than one @code{\begin} or so.

While all operations above are to kill `containers' which surround some
text, universal argument (@kbd{C-u}) for these commands kills not only
`containers' but also `contents' of them.  See below as a sample.

@example
Original text:                  [prefix] k      C-u [prefix] k
Main \footnote@{note@} here.    Main note here. Main  here.
       ~(cursor)
@end example

@node Filling, Updation of includeonly, Changing and Deleting, Top
@comment  node-name,  next,  previous,  up
@chapter Filling
@cindex filling

@section Filling an item
@cindex filling an item
@cindex prefix i

  To fill a term (descriptive sentences) of @code{\item}, type

@c @table @kbd
@c @item [prefix] i
@c         @dots{} fill item
@c @end table
@table @kbd
@item M-q
        @dots{} fill item
@end table

@noindent
on that item.

  YaTeX uses the value of  the variable @code{YaTeX-item-regexp} as the
regular expression to search item header  in  itemize environment.
If you make a newcommand to itemize terms(e.g. @code{\underlineitem}), put

@lisp
        (setq YaTeX-item-regexp
              "\\(\\\\\\(sub\\)*item\\)\\|\\(\\\\underlineitem\\)")
@end lisp
@cindex YaTeX-item-regexp

in your @file{~/.emacs}.  If you are not familiar with regular expression
for Emacs-Lisp, name  a newcommand  for  `itemize' beginning  with
@code{\item} such as @code{\itembf}, not @code{\bfitem}.

  This function reformats the @code{\item} into `hang-indented' style.
For example:

@example
itemize, enumerate environment:
       >
       >\item[foo] `foo' is the typical word for describing an
       >           arbitrarily written....
description environment:
       > \item[bar] When the word `for' is used as an arbitrarily
       >        word, `bar'  is bound to follow it.
@end example

  Note that the indent depth of an @code{\item} word and its descriptive
paragraph are the same in latter case.  If you want to use different
depth, invoke fill-paragraph at the beginning of non-whitespace
character(see below).

@section Filling paragraph
@cindex Filling paragraph
@cindex M-q

  Fill-paragraph is little bit adapted for La@TeX{} sources.  It retains from
filling in certain environments where formatting leads to a disaster such
as verbatim, tabular, or so.  And it protects @code{\verb} expressions
from being folded (The variable @code{YaTeX-verb-regexp} controls this).
Besides, putting cursor on the first occurrence of non-whitespace
character on a line changes the fill-prefix temporarily to the depth of
the line.

@node Updation of includeonly, What column, Filling, Top
@comment  node-name,  next,  previous,  up
@chapter Updation of @code{\includeonly}
@cindex includeonly

  When you edit splitting source texts, the notation

@example
        \includeonly@{CurrentEditingFileName@}
@end example

@noindent
in the main file reduces the time of typesetting.  If you want
to hack other file a little however, you have to rewrite it to

@example
        \includeonly@{OtherFileNameYouWantToFix@}
@end example

@noindent
in the main file.  YaTeX automatically detects that the current
edited text is not in includeonly list and prompts you

@example
        A)dd R)eplace %)comment?
@end example

in the minibuffer.  Type @kbd{a} if you want to add the current file name
to @code{\includeonly} list, @kbd{r} to replace \@code{includeonly} list
with the current file, and type @kbd{%} to comment out the
@code{\includeonly} line.

@node What column, Intelligent newline, Updation of includeonly, Top
@comment  node-name,  next,  previous,  up
@chapter What column?
@cindex what column
@cindex complex tabular
@cindex prefix &

  We  are often get  tired of  finding the corresponding column in
large tabulars.  For example,

@example
        \begin@{tabular@}@{|c|c|c|c|c|c|c|c|@}\hline
         Name&Position&Post No.&Addr.&Phone No.&FAX No.&
                Home Addr.&Home Phone\\ \hline
         Thunder Bird & 6 & 223 & LA & xxx-yyy &
          zzz-www & Japan & 9876-54321 \\
           & 2 & \multicolumn@{2@}@{c|@}@{Unknown@}
                &&&(???)
         \\ \hline
         \end@{tabular@}
@end example

Suppose you have the cursor located  at @code{(???)} mark, can you tell
which  column it is  belonging  at once?  Maybe no.  In such case,
type

@table @kbd
@item [prefix] &
        @dots{} What column
@end table

@noindent
in  that position.   YaTeX  tells you  the  column header  of  the
current  field.  Since  YaTeX  assumes  the  first line of tabular
environment  as a row of column  headers, you  can  create a row of
virtual column  headers by  putting  them  in the  first line  and
commenting that line with @code{%}.

@node Intelligent newline, Usepackage checker, What column, Top
@comment  node-name,  next,  previous,  up
@chapter Intelligent newline
@cindex Intelligent newline
@cindex ESC RET
@cindex M-C-m

  At the end of begin-type completion of tabular[*], array, itemize,
enumerate or tabbing environment, or typing 

@table @kbd
@item ESC RET
        @dots{} Intelligent newline
@end table

@noindent 
in these environments inserts the contents corresponding to the current
environment in the next line.  (At the begin-type completion, this
contents can be removed by `undo'.)  In @code{tabular} environment, for
example, @kbd{ESC RET} inserts the certain number of @code{&} and trailing
@code{\\}, and @code{\hline} if other @code{\hline} is found in backward.
Here are the list of contents vs. environments.

@itemize
@item @code{tabular}, @code{tabular*}, @code{array}

        Corresponding number of @code{&} and  @code{\\}.
        And @code{\hline} if needed.

@item @code{tabbing}

        The same number of @code{\>} as @code{\=} in the first line.

@item @code{itemize}, @code{enumerate}, @code{description}, @code{list}

        @code{\item} or @code{item[]}.
@end itemize

  Note that since this function works seeing the contents of the first
line, please call this after the second line if possible.

  If you want to apply these trick to other environments, @code{foo}
environment for example, define the function named
@code{YaTeX-intelligent-newline-foo} to insert corresponding contents.
That function will be called at the beginning of the next line after the
newline is inserted to the current line.  Since the function
@code{YaTeX-indent-line} is designed to indent the current line properly,
calling this function before your code to insert certain contents must be
useful.  See the definition of the function
@code{YaTeX-intelligent-newline-itemize} as an example.

@node Usepackage checker, Online help, Intelligent newline, Top
@comment  node-name,  next,  previous,  up
@chapter Usepackage checker
@cindex usepackage

When you input begint-type, section-type, maketitle-type macros with
completion, and it requires some LaTeX2e package, YaTeX examines
the existence of correct @code{\usepackage}.  If not, YaTeX inserts
the @code{\usepackage@{@}} declaration corresponding to input macro.

To activate the package completion for your favarite package,
set the variable @code{YaTeX-package-alist-private} correctly.
Please refere the value of @code{YaTeX-package-alist-default} as an
example.

@node Online help, Browsing file hierarchy, Usepackage checker, Top
@comment  node-name,  next,  previous,  up
@chapter Online help
@cindex online help
@cindex prefix ?
@cindex prefix /
@cindex apropos
@cindex keyword search

  YaTeX provides you the online help with popular La@TeX{} commands.

  Here are the key strokes for the online help.

@table @kbd
@item [prefix] ?
        @dots{} Online help
@item [prefix] /
        @dots{} Online apropos
@end table

@section Online help

  `Online help' shows the documentation for the popular La@TeX{}
commands(defaults to the commands on the cursor) in the next buffer.
There are two help file, `global help' and `private help'.  The former
file contains the descriptions on the standard La@TeX{} command and is
specified its name by variable @code{YaTeX-help-file}.  Usually, the
global help file should be located in public space (@code{$EMACSEXECPATH}
by default) and should be world writable so that anyone can update it to
enrich its contents.  The latter file contains descriptions on
non-standard or personal command definitions and is specified by
@code{YaTeX-help-file-private}.  This file should be put into private
directory.

@section Online apropos

  `Online  apropos' is an  equivalent  of GNU Emacs's apropos.  It
shows all the documentations that contains  the keyword entered by
the user.

@section When no descriptions are found...

  If there is no description  on a command   in help files,  YaTeX
requires you to  write a description on  that command.  If you are
willing  to  do, determine  which help file  to add and  write the
description on it referring  your manual of (La)TeX.  Please  send
me your additional descriptions if you  describe  the help on some
standard commands.   I might  want    to include it  in   the next
distribution.

@node Browsing file hierarchy, Cooperation with other packages, Online help, Top
@comment  node-name,  next,  previous,  up
@chapter Browsing file hierarchy
@cindex hierarchy
@cindex browsing

  When you are editing multi-file source, typing

@table @kbd
@item [prefix] d
        @dots{} browse file hierarchy
@end table

@noindent
asks you the parent-most file (which may be defaulted) and displays the
documentation hierarchy in the next window.  In this buffer, the following
commands are available.

@table @kbd
@item n
        @dots{} move to the next line and show its contents
@item p
        @dots{} move to the previous line and show its contents
@item N
        @dots{} move to the next file in the same inclusion level
@item P
        @dots{} move to the previous file in the same inclusion level
@item j
        @dots{} move to the next line
@item k
        @dots{} move to the previous line
@item u
        @dots{} move to the parent file
@item .
        @dots{} show the current files contents in the next window
@item SPC
        @dots{} scroll up the current file window
@item DEL, b
        @dots{} scroll down the current file window
@item <
        @dots{} show the beginning of the current file
@item >
        @dots{} show the end of the current file
@item >
        @dots{} return to the previous postion after @kbd{<} or @kbd{>}
@item RET, g
        @dots{} open the current file in the next window
@item mouse-2
        @dots{} same as RET(available only with window system)
@item o
        @dots{} other window
@item 1
        @dots{} delete other windows
@item -
        @dots{} shrink hierarchy buffer window
@item +
        @dots{} enlarge hierarchy buffer window
@item ?
        @dots{} describe mode
@item q
        @dots{} quit
@end table

  Note that operations on the file contents in the next window do not work 
correctly when you close the corresponding file.

@node Cooperation with other packages, Customizations, Browsing file hierarchy, Top
@comment  node-name,  next,  previous,  up
@chapter Cooperation with other packages

  YaTeX works better with other brilliant packages.

@section gmhist
@cindex gmhist
@cindex command history
@cindex minibuffer history

  When you are loading @file{gmhist.el} and @file{gmhist-mh.el}, you can
use independent command history list at the prompt of preview command
(@kbd{[prefix] tp}) and print command (@kbd{[prefix] tl}).  On each
prompt, you can enter the previous command line string repeatedly by
typing @kbd{M-p}.

@section min-out
@cindex min-out

  @file{min-out}, the outline minor mode,  can  be used in yatex-mode
buffers.  If you want to use it with YaTeX, please refer the
file @file{yatexm-o.el} as an example.

@node Customizations, Etcetera, Cooperation with other packages, Top
@comment  node-name,  next,  previous,  up
@chapter Customizations
@cindex customizations

  You can customize YaTeX by setting Emacs-Lisp variables and by making
add-in functions.

@menu
* Lisp variables::              
* Add-in functions::            
* Add-in generator::            
@end menu

@node Lisp variables, Add-in functions, Customizations, Customizations
@comment  node-name,  next,  previous,  up
@section Lisp variables
@cindex customizable variables

  You can change the key assignments or make completion more comfortable
by setting the values of various variables which control the movement of
yatex-mode.

  For example, if you want to change the prefix key stroke from @kbd{C-c}
to any other sequence, set YaTeX-prefix to whatever you want to use.  If
you don't want to use the key sequence @kbd{C-c letter} which is assumed
to be the user reserved sequence in Emacs world, set
@code{YaTeX-inhibit-prefix-letter} to @code{t}, and all of the default key
bind of @kbd{C-c letter} will turn to the corresponding @kbd{C-c C-letter}
(but the region based completions that is invoked with @kbd{C-c
Capital-letter} remain valid, if you want to disable those bindings, set
that variable to 1 instead of @code{t}).

@menu
* All customizable variables::  
* Sample definitions::          
* Hook variables::              
* Hook file::                   
@end menu

@node All customizable variables, Sample definitions, Lisp variables, Lisp variables
@comment  node-name,  next,  previous,  up
@subsection All customizable variables
@cindex all customizable variables

  Here are the customizable variables of yatex-mode.  Each value setq-ed
in @file{~/.emacs} is preferred and that of defined in @file{yatex.el} is
neglected.  Parenthesized contents stands for the default value.  When you
are to change some of these variables,  see more detailed documentation of
the variable by @kbd{M-x describe-variable}.

@defvar YaTeX-japan
Set this nil to produce all messages in English
(@code{Depends on Japanese feature of Emacs})
@end defvar

@defvar YaTeX-kanji-code
Default buffer-file-coding-system for YaTeX modes' buffer.
Set this 0 to no language conversion.  Nil to preserve original 
coding-system. 1=Shift JIS, 2=JIS, 3=EUC, 4=UTF-8 (@code{1 or 2})
@end defvar

@defvar YaTeX-prefix
Prefix key stroke (@kbd{C-c})
@end defvar

@defvar YaTeX-inhibit-prefix-letter
Change key stroke from @kbd{C-c letter} to @kbd{C-c C-letter} (@code{nil})
@end defvar

@defvar YaTeX-fill-prefix
Fill-prefix used in yatex-mode (@code{nil})
@end defvar

@defvar YaTeX-user-completion-table
Name of user dictionary where learned completion table will be stored. 
(@code{"~/.yatexrc"})
@end defvar

@defvar tex-command
La@TeX{} typesetter command (@code{"latex"})
@end defvar

@defvar dvi2-command
Preview command (@code{"xdvi -geo +0+0 -s 4"})
@end defvar

@defvar dviprint-command-format
Command format to print dvi file (@code{"dvi2ps %f %t %s | lpr"})
@end defvar

@defvar dviprint-from-format
Start page format of above %f. %b will turn to start page (@code{"-f %b"})
@end defvar

@defvar dviprint-to-format
End page format of above %t. %e will turn to @code{end} page (@code{"-t %e"})
@end defvar

@defvar makeindex-command
Default makeindex command (@code{"makeindex"} (@code{"makeind"} on MS-DOS))
@end defvar

@defvar YaTeX-dvipdf-command
Default command name to convert .dvi to PDF (@code{"dvipdfmx"})
@end defvar

@defvar YaTeX-need-nonstop
Put @code{\nonstopmode@{@}} or not (@code{nil})
@end defvar

@defvar latex-warning-regexp
Regular expression of warning message latex command puts out 
(@code{"line.* [0-9]*"})
@end defvar

@defvar latex-error-regexp
Regular expression of error message (@code{"l\\.[1-9][0-9]*"})
@end defvar

@defvar latex-dos-emergency-message
Message latex command running on DOS puts at abort (@code{"Emergency stop"})
@end defvar

@defvar YaTeX-item-regexp
Regular expression of item command (@code{"\\\\item"})
@end defvar

@defvar YaTeX-verb-regexp
Regexp of verb family.  Omit \\\\. (@code{"verb\\*?\\|path"})
@end defvar

@defvar YaTeX-nervous
T for using local dictionary (@code{t})
@end defvar

@defvar YaTeX-sectioning-regexp
Regexp of La@TeX{} sectioning command 
(@code{"\\(part\\|chapter\\*?\\|\\(sub\\)*\\(section\\|paragraph\\)\\*?\\)\\b"})
@end defvar

@defvar YaTeX-fill-inhibit-environments
Inhibit fill in these environments 
(@code{'("tabular" "tabular*" "array" "picture" "eqnarray" "eqnarray*" "equation" "math" "displaymath" "verbatim" "verbatim*")})
@end defvar

@defvar YaTeX-uncomment-once
T for deleting all preceding @code{%} (@code{nil})
@end defvar

@defvar YaTeX-close-paren-always
T for always close all parenthesis automatically, @code{nil} for only eol 
(@code{t})
@end defvar

@defvar YaTeX-auto-math-mode
Switch math-mode automatically (@code{t})
@end defvar

@defvar YaTeX-math-key-list-private
User defined alist, math-mode-prefix vs completion alist
used in image completion (@code{nil}).  See @file{yatexmth.el}
for the information about how to define a completion alist.
@end defvar

@defvar YaTeX-default-pop-window-height
Initial height of typesetting buffer when one-window.
Number for the lines of the buffer, numerical string for
the percentage of the screen-height. @code{nil} for half height (10)
@end defvar

@defvar YaTeX-help-file
Global online help file name (@file{$doc-directory/../../site-lisp/YATEXHLP.eng})
@end defvar

@defvar YaTeX-help-file-private
Private online help file name (@file{"~/YATEXHLP.eng"})
@end defvar

@defvar YaTeX-no-begend-shortcut
Disable [prefix] b ?? shortcut (@code{nil)}
@end defvar

@defvar YaTeX-hilit-pattern-adjustment-private
List of the list that contain the regular expression and the symbol of
logical meaning of the string that matches the pattern.  See also the
value from @code{(assq 'yatex-mode hilit-patterns-alist)} and the value of 
@code{YaTeX-hilit-pattern-adjustment-default} (and even the document of
hilit19.el).
@end defvar

@defvar YaTeX-sectioning-level
Alist of LaTeX's sectioning command vs its height.
@end defvar

@defvar YaTeX-hierarchy-ignore-heading-regexp
@code{YaTeX-display-hierarchy} searches for sectioning command first, and
comment line secondary as a file headings.  In latter case, ignore lines
that match with regular expression of this variable.  Default value of
this variable is RCS header expressions and mode specifying line `-*- xxxx 
-*'.
@end defvar

@defvar YaTeX-skip-default-reader
Non-nil for this variable skips the default argument reader of
section-type command when add-in function for it is not defined 
(@code{nil})
@end defvar

@defvar YaTeX-create-file-prefix-g
When typing @kbd{prefix g} on the @code{\include} line,
open the target file even if the file doesn't exist (@code{nil})
@end defvar

@defvar YaTeX-simple-messages
Simplyfy messages of various completions (@code{nil})
@end defvar

@defvar YaTeX-hilit-sectioning-face
When hilit19 and yatex19 is active, YaTeX colors the sectioning commands.
This variable specifies the foreground and background color of
@code{\part} macro.  The default value is @code{'(yellow/dodgerblue
yellow/slateblue)}.  The first element of this list is for the screen when
@code{hilit-background-mode} is @code{'light}, and the second element is
for @code{'dark}.  You should specify both color as `forecolor/backcolor'.
@end defvar

@defvar YaTeX-hilit-sectioning-attenuation-rate
When color mode, this variable specifies how much attenuate the color
density of @code{\subparagraph} compared with that of @code{\chapter} 
(@code{'(15 40)}) See also @code{YaTeX-hilit-sectioning-face}.
@end defvar

@defvar YaTeX-use-AMS-LaTeX
If you use AMS-LaTeX, set to @code{t} (@code{nil})
@end defvar

@defvar YaTeX-use-LaTeX2e
If you use LaTeX2e, set to @code{t} (@code{t})
@end defvar

@defvar YaTeX-template-file
File name which is automatically inserted at creation
(@code{~/work/template.tex})
@end defvar

@defvar YaTeX-search-file-from-top-directory
Non-nil means to search input-files from the directory where main file exists
(@code{t})
@end defvar

@defvar YaTeX-use-font-lock
Use font-lock to fontify buffer or not (@code{(featurep 'font-lock)}
@end defvar

@defvar YaTeX-use-hilit19
Use hilit19 to highlight buffer or not (@code{(featurep 'hilit19)}
@end defvar

@defvar YaTeX-use-italic-bold
YaTeX tries to search italic, bold fontsets or not
(@code{t} if Emacs-20 or later).  This variable is effective only when
font-lock is used.
(@code{(featurep 'hilit19)}
@end defvar

@defvar YaTeX-singlecmd-suffix
Suffix which is always inserted after maketitle-type macros.
@code{"{}"} is recommended.
@end defvar

@defvar YaTeX-package-alist-private
Alist of LaTeX2e-package name vs. lists of macros in it.
Set this alist properly and YaTeX automatically check the declaratiion of
`usepackage' for corresponding macro, when you input that macro with
completion.  If required `usepackage' is not found, YaTeX also
automatically inserts `\usepackage'.  Alist is as follows;
@lisp
   '((PackageName1
        (completionType ListOfMacro)
        (completionType ListOfMacro))
     (PackageName2
        (completionType ListOfMacro)
        (completionType ListOfMacro...))....)
@end lisp
completionType is one of @code{env, section, maketitle}.
Consult the value of @code{YaTeX-package-alist-default} as an example.
@end defvar

@defvar YaTeX-tabular-indentation
At indentation by @kbd{C-i} in tabular or array environment, 
YaTeX put the additional spaces to the normail indentation depth.
The number of additional spaces is the product of YaTeX-tabular-indentation
and the number of column position in tabular.
@end defvar

@defvar YaTeX-noindent-env-regexp
Regexp of environment names that should begin with no indentation.
All verbatime-like environment name should match with.
@end defvar

@defvar YaTeX-ref-default-label-string
Default \\ref time string format.
This format is like strftime(3) but allowed conversion char are as follows;
%y -> Last 2 digit of year,  %b -> Month name,  %m -> Monthe number(1-12),
%d -> Day,  %H -> Hour,  %M -> Minute,  %S -> Second,
%qx -> alphabetical-decimal conversion of yymmdd.
%qX -> alphabetical-decimal conversion of HHMMSS.
Beware defualt label-string should be always unique.  So this format string
should have both time part (%H+%M+%S or %qX) and date
part (%y+(%b|%m)+%d or %qx).
@end defvar

@defvar YaTeX-ref-generate-label-function
Function to generate default label string for unnamed \\label{}s.
The function pointed to this value should take two arguments.
First argument is LaTeX macro's name, second is macro's argument.
Here is an example for using this value.
@lisp
  (setq YaTeX-ref-generate-label-function 'my-yatex-generate-label)
  (defun my-yatex-generate-label (command value)
    (and (string= command "caption")
         (re-search-backward "\\\\begin{\\(figure\\|table\\)}" nil t)
         (setq command (match-string 1)))
    (let ((alist '(("chapter" . "chap")
                   ("section" . "sec")
                   ("subsection" . "subsec")
                   ("figure" . "fig")
                   ("table" . "tbl"))))
      (if (setq command (cdr (assoc command alist)))
          (concat command ":" value)
        (YaTeX::ref-generate-label nil nil))))
@end lisp
@end defvar


@node Sample definitions, Hook variables, All customizable variables, Lisp variables
@comment  node-name,  next,  previous,  up
@subsection Sample definitions
@cindex prefix key stroke
@cindex fill-prefix

 For instance, to change the prefix key stroke to @kbd{ESC}, and name of
the user dictionary @file{~/src/emacs/yatexrc}, and set @code{fill-prefix}
to single TAB character, add the following @code{setq} to @file{~/.emacs}.

@lisp
        (setq YaTeX-prefix "\e"
              YaTeX-user-completion-table "~/src/emacs/yatexrc"
              YaTeX-fill-prefix "       ")
@end lisp

@node Hook variables, Hook file, Sample definitions, Lisp variables
@comment  node-name,  next,  previous,  up
@subsection Hook variables
@cindex hook variables

  More customizations will be done by the hook-function defined in
hook-variable @code{yatex-mode-hook}.  This is useful to define a shortcut
key sequence to enter some environments other than @code{document} and
@code{enumerate} etc.  The following statement defines @code{[prefix] ba}
to enter @code{\begin@{abstract@}} ...  @code{=end@{abstract@}}
immediately.

@lisp
        (setq yatex-mode-hook
              '(lambda() (YaTeX-define-begend-key "ba" "abstract")))
@end lisp

        You should use functions @code{YaTeX-define-key}, or
@code{YaTeX-define-begend-key}  to define all the key sequences of
yatex-mode.

@node Hook file,  , Hook variables, Lisp variables
@comment  node-name,  next,  previous,  up
@subsection Hook file
@cindex hook file

  You can stuff all of YaTeX related expressions into a file named
@file{yatexhks.el} if you have a lot of codes.  YaTeX automatically load
this file at the initialization of itself.  Using @file{yatexhks.el}
makes @code{yatex-mode-load-hook} unnecessary.

@node Add-in functions, Add-in generator, Lisp variables, Customizations
@comment  node-name,  next,  previous,  up
@section Add-in functions
@cindex add-in functions
@cindex yatexadd.el

  You can easily  define  a function to input  detailed  arguments
with completion according  to La@TeX{} environments  or commands. 

@c @node What is add-in functions?,  , Add-in functions, Add-in functions
@comment  node-name,  next,  previous,  up
@subsection What is add-in functions?
@cindex tabular

  When you input @code{tabular} environment, don't you think ``I want
YaTeX to complete its argument toward my favorite one such as
@code{@{|c|c|c|@}}...''?  Yes, you can define the function to complete
arguments for any environment and any La@TeX{} commands.

@subsection Procedure

  Here is the procedure to define add-in functions.
@enumerate
@item
Define the function
@item
Put the function into @file{yatexhks.el}
@end enumerate

@menu
* How the add-in function works::  
* How the function is called::  
* Useful functions for creating add-in::  
* Contribution::                
@end menu

@node How the add-in function works, How the function is called, Add-in functions, Add-in functions
@comment  node-name,  next,  previous,  up
@subsection How the add-in function works

There are three types of add-in.

@enumerate
@item
Option add-in
@item
argument add-in
@item
enclosing add-in
@end enumerate

@dfn{Option add-in} returns the
La@TeX{}'s optional parameters such as optional strings after
@code{\begin@{ENV@}}, optional strings between a section-type command
and its first argument, and optional strings just after type
maketitle-type command.  The following illustrates the name of add-in
functions, where underlined strings are generated by add-in functions.

@display
\begin@{table@}[ht]		(Function name: YaTeX:table)
             ~~~~
\put(100,200)@{@}		(Function name: YaTeX:put)
    ~~~~~~~~~
\sum_@{i=0@}^@{n@}		(Function name: YaTeX:sum)
    ~~~~~~~~~~
@end display

  Obviously, the function name is decided by concatenating the prefix
`YaTeX:' and La@TeX{} command's name.

  Another add-in type is @dfn{argument add-in}, which completes arguments
for section-type commands.

@display
\newcommand@{\foo@}@{bar@}	(Function name: YaTeX::newcommand)
            ~~~~  ~~~
@end display

  When the section-type command is inputted, the function named by
concatenating `YaTeX::' and section-type command, is called automatically
with an integer argument which indicates which argument of section-type
command is being read.  Thus the add-in should determine the
job referring the value of its argument.

  @dfn{enclosing add-in} is for modifying and/or checking the region that
will be enclosed by section-type commands via @kbd{[prefix] S}.  An
enclosing add-in function will be called with two arguments, beginning of
the enclosed region and end of the region.  Suppose you want to enclose
the existing text @code{(a+b)/c} by @code{\frac{}}.

@display
a/c
|  |
A  B
@end display

You do set-mark-command at point A and then move to point B.  Typing
@kbd{[prefix] S} and input @code{frac} enclose the region like this;

@display
\frac{a/c}
@end display

Normally, the expression @code{a/c} is translated to
@code{\frac@{a@}@{c@}}. An enclosing add-in is useful for modifying
@code{/} to @code{@}@{}.

@menu
* Defining option-add-in::      
* Defining argument-add-in::    
* Defining enclosing-add-in::   
@end menu

@node Defining option-add-in, Defining argument-add-in, How the add-in function works, How the add-in function works
@comment  node-name,  next,  previous,  up
@subsubsection Defining `option add-in'

  If you want @code{@{|c|c|c|@}} for all @code{tabular} environment,

@lisp
        (defun YaTeX:tabular ()
          "@{|c|c|c|@}")
@end lisp

@noindent
is enough.  If you want more complicated format, define as below.

@lisp
        (defun YaTeX:tabular ()
          "@{@@@{\\vrule width 1pt\\ @}|||@@@{\\ \\vrule width 1pt@}@}")
@end lisp

@noindent
Note that the character @code{\} must be described as @code{\\} in
Emacs-Lisp.  The next example reads the tabular format from keyboard.
@lisp
        (defun YaTeX:tabular ()
          (concat "@{" (read-string "Rule: ") "@}"))
@end lisp

@node Defining argument-add-in, Defining enclosing-add-in, Defining option-add-in, How the add-in function works
@comment  node-name,  next,  previous,  up
@subsubsection Defining `argument add-in'

  This section describes how to define the add-in function for
@code{\newcommand}.

  The first argument of @code{\newcommand} begins always with @code{\}.
The second argument is usually so complex that we can not edit them in the 
minibuffer.  Here is the created function considering this.

@lisp
        (defun YaTeX::newcommand (n)	;n is argument position
          (cond
           ((= n 1)			;1st argument is macro name
            (read-string "Command: " "\\")) ;initial input `\' 
           ((= n 2) "")			;do nothing when reading arg#2
           (t nil)))
@end lisp

  Note that when the `argument add-in' function return `nil', normal
argument reader will be called.

@node Defining enclosing-add-in,  , Defining argument-add-in, How the add-in function works
@comment  node-name,  next,  previous,  up
@subsubsection Defining `enclosing add-in'

  This section describes how to define the add-in function for
text enclosed by @code{\frac@{@}}.

  When enclosing the text @code{5/3} by @code{\frac@{@}}, you might want to
replace @code{/} with @code{@}@{}.  Enclosing function
@code{YaTeX::frac-region} is called with two arguments, beginning of
enclosed text and end of enclosed text.  The function is expected to
replace @code{/} with @code{@}@{}.  Here is an example expression.

@lisp
(defun YaTeX::frac-region (beg end)
  (catch 'done
    (while (search-forward "/" end t)
      (goto-char (match-beginning 0))
      (if (y-or-n-p "Replace this slash(/) with `}{'")
	  (throw 'done (replace-match "}{")))
      (goto-char (match-end 0)))))
@end lisp

@node How the function is called, Useful functions for creating add-in, How the add-in function works, Add-in functions
@comment  node-name,  next,  previous,  up
@subsection How the function is called

  YaTeX calls the add-in functions for specified begin-type, section-type,
and maketitle-type command, if any.  `Option add-in' functions for
begin-type are called when @code{\begin@{ENV@}} has been inserted,
functions for section-type are called just before input of the first
argument, and functions for maketitle-type is called after maketitle-type
command has been inserted.  `Argument add-in' functions are called at each
entry of arguments for section-type commands.

@node Useful functions for creating add-in, Contribution, How the function is called, Add-in functions
@comment  node-name,  next,  previous,  up
@subsection Useful functions for creating add-in

  Many add-in functions for typical La@TeX{} commands are defined in
@file{yatexadd.el}.  Those are also useful as references.  Here are the
short descriptions on useful functions, where [F] means function, [A]
means arguments, [D] means description.

@table @kbd
@item [F]
YaTeX:read-position
@itemx [A]
Character list which can show up in the brackets
@itemx [D]
   Return the location specifier such as `[htb]'.  When
nothing is entered, omit [] itself.  If the possible characters
are "htbp", call this function as
@code{(YaTeX:read-position "htbp")}

@item [F]
YaTeX:read-coordinates
@itemx [A]
Base prompt, X-axis prompt, Y-axis prompt     (each optional)        
@itemx [D]
  Read the coordinates with the prompt ``BasePrompt X-axisPrompt:'' for
X-axis, ``BasePrompt Y-axisPrompt:'' for Y-axis, and return it in the form 
of ``(X,Y)''.  The default prompts are @code{Dimension}, @code{X},
@code{Y} respectively.

@item [F]
YaTeX:check-completion-type
@itemx [A]
One of the symbols: 'begin, 'section, or 'maketitle
@itemx [D]
  Check the current completion type is specified one and cause error if
not. The variable @code{YaTeX-current-completion-type} holds the symbol
according to the current completion type.
@end table

@node Contribution,  , Useful functions for creating add-in, Add-in functions
@comment  node-name,  next,  previous,  up
@subsection Contribution

  If you make your own pretty function and you let it be in public, please 
send me the function.  I'm going to include it in the next release.

@node Add-in generator,  , Add-in functions, Customizations
@comment  node-name,  next,  previous,  up
@section Add-in generator

  First, don't forget to read the section of add-in functions @ref{Add-in
functions}.  If you easily understand how to define them, there's no need
to read this section.  But being not familiar with Emacs-Lisp, when you
don't have clear idea what to do, this section describes how to get YaTeX
make add-in function.

  There are two methods of generation.  One is for fully interactive
generator for beginners and another requires little knowledge of
Emacs-Lisp.

@subsection Generator for beginners
  The former generator is called by
@center @kbd{M-x YaTeX-generate}

@noindent
strokes.  All you have to do is follow the guidances.  Defying them may cases 
the disaster (I wonder what is it???).  So when you make some mistake, it
is recommendable to type @kbd{C-g} and start afresh.

@subsection Simple generator

  The latter generator is invoked by the next sequence.
@center @kbd{M-x YaTeX-generate-simple}
This generator can make both ``option add-in'' and ``argument add-in''
(@emph{refer the section add-in functions}
@ref{How the add-in function works}), whereas @code{YaTeX-generate}
cannot make ``argument addin''.

  For example, assume you have the LaTeX command as follows.

@example
	\epsinput[t](250,50)@{hoge.eps@}@{plain@}@{Picture of foo@}
	         (A)  (B)     (1)      (2)      (3)
	(A)Optional parameter to specify the position
	   One of t(top), b(bottom), l(left), r(right)
	(B)Maximum size of frame
	(1)1st argument is filename of EPS file
	(2)2nd argument indicates
		plain		do nothing
		frame		make frame around image
		dframe		make double-frame around image
	   for included EPS file.
	(3)Caption for the picture
@end example

  Now get start with generation.  Typing @kbd{M-x YaTeX-generate-simple}
brings the prompt:
@display
                (O)ption? (A)rgument?
@end display

@subsubsection Generating ``option add-in''
@cindex option add-in

  Since (A), (B) above are optional argument, all we have to do to
complete them is define the option add-in for them.  Let's generate the
function to complete (A).

@display
                M-x YaTeX-generate-simple RET
                epsinput RET
                o
@end display

@noindent
Typing as above leads the next prompt.

@display
Read type(1): (S)tring (C)omplete (F)ile ([)option (P)osition co(O)rd. (q)uit
@end display

@noindent
  This asks that ``Which type is the completion style of 1st argument?''.
Here are the possible completion style.

@table @code
@item String
read plain string
@item Complete
read with completion
@item File
read file name
@item Option
read optional string (if string omitted, omit [] too)
@item Position
read positional option (like [htbp])
@item Coord.
read coordinates
@item Quit
quit from generating
@end table

  Since (A) is the optional argument to specify the location of included
EPS file, the completion style is @code{Position}, and the possible
characters are t, b, l, and r.  To tell these information to generator,
operate as follows.

@display
                Read type(1).... 		p
                Acceptable characters:		tblr RET
@end display

  (B) is coordinate.  So its completion style is coOrd.  We want a prompt
meaning ``Maximum size'' when completion.

@display
                Read type(2)....		o
                Prompt for coordinates:		Max size RET
@end display

  That's all for optional argument.  Select quit.

@display
                Read type(3)....		q
@end display

  Then the generated option add-in function for \epsinput will be shown in
the next window.

@subsubsection Generating ``argument add-in''
@cindex argument add-in

  Next, create the argument add-in.  The arguments for \epsinput are EPS
file name, framing style, and caption string in sequence.

@display
                M-x YaTeX-generate-simple RET
                epsinput RET
                a
@end display

  Above key strokes bring the prompt that asks the number of argument.
Answer it with 3.

@display
                How many arguments?: 3 RET
@end display

  Then the generator asks the completion style and prompt for completion.
Answer them.  @kbd{f} for FileName and prompt string.

@display
                Read type(1)....		f
                Prompt for argument#1		EPS file name RET
@end display

  The second argument is one of selected symbol.  So the completion type
is @code{Completion}.

@display
                Read type(2)....		c
                Prompt for argument#2		Include style RET
@end display

  Then all the candidates ready to be read.  Type single RET after
entering all.

@display
		Item[1](RET to exit):		plain RET
		Item[2](RET to exit):		frame RET
		Item[3](RET to exit):		dframe RET
		Item[4](RET to exit):		RET
@end display

  The following prompt asks whether the entered string must belong to
candidates or not.  In this case, since the argument must be one of
@code{plain}, @code{frame}, and @code{dframe}, type @code{y}.

@display
                Require match? (y or n)		y
@end display

  The last argument is the caption string for which any completion is
needed.

@display
                Read type(3)....		s
                Prompt for argument#3		Caption RET
                default:			Figure of RET
@end display

  Finally we'll get the argument add-in in the next window.

@subsection Contribution

  If you get your own pretty function and you let it be in public, please 
steel yourself in the happy atmosphere and do not send me the function.
I do know it is not fine because it is generated by yatexgen:-p.

@node Etcetera, Copying, Customizations, Top
@comment  node-name,  next,  previous,  up
@chapter Etcetera

  The standard completion  tables provided  in @file{yatex.el} contain  a
few La@TeX{}  commands  I frequently use.  This is  to lessen the key
strokes to  complete  entire  word, because   too  many candidates
rarely used  often cause too many  hits.   Therefore always try to
use completion  in order to  enrich your dictionary,  and you will
also find `Wild Bird' growing suitable for your La@TeX{} style.

  The package name `Wild Bird' is the English translation of Japanese
title `Yachou', which is a trick on words of Japanese.

@node Copying,  , Etcetera, Top
@comment  node-name,  next,  previous,  up
@chapter Copying

  This   program   is  distributed   as   a   free   software.   You   can
use/copy/modify/redistribute this software freely  but with NO warranty to
anything  as a result  of using  this software.   Adopting code  from this
program is  also free.  But  I would not  do contract act.

Any reports  and suggestions are  welcome as long  as I feel  interests in
this software.  My possible  e-mail address is `yuuji@@yatex.org'.  (as of
Jan.2004)  And there  is  mailing  list for  YaTeX.   Although the  common
language is Japanese,  questions in English will be  welcome.  To join the
ML,   send  the   mail  whose   subject   is  `append'   to  the   address
`yatex@@yatex.org.    If   you  have   some   question,   please  ask   to
`yatex-admin@@yatex.org'.

  The specification of this software will be surely modified
(depending on my feelings) without notice :-p.


@flushright
                                                        HIROSE Yuuji
@end flushright
@bye

Local variables:
mode: texinfo
fill-prefix: nil
fill-column: 74
End:
